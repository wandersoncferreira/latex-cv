\documentclass[10pt]{article}
\usepackage[margin=3cm]{geometry}
\usepackage[pdftex]{hyperref}
\usepackage[brazilian]{babel}
\usepackage[utf8]{inputenc}
\usepackage[T1]{fontenc}
\usepackage{bibentry}

\usepackage{array, xcolor}
\definecolor{lightgray}{gray}{0.8}

\newcolumntype{L}{>{\raggedleft}p{0.2\textwidth}}
\newcolumntype{R}{p{0.8\textwidth}}
\newcommand\VRule{\color{lightgray}\vrule width 1.5pt}

\title{\bfseries\Huge Wanderson Ferreira}
\author{iagwanderson@gmail.com \\
\href{http://wandersoncferreira.github.io}{http://wandersoncferreira.github.io}}
\date{\today}
\begin{document}
\maketitle

% Section for the address
\begin{center}
  \begin{minipage}{.7\textwidth}
    Address: Av. Jaguaré, 249, Butantã\\
    São Paulo 05643-000\\
    São Paulo, Brazil.
  \end{minipage}%
  \begin{minipage}{.5\textwidth}
    Nationality: Brazilian\\
    Birthday: November 13th, 1991\\
    Mobile: +55 11 966428772
  \end{minipage}
\end{center}

\section*{Objectives}
A position as Machine Learning Engineer. Since started to work with ML, I've been studying a very broad range of applications related to Artificial Intelligence. I've been leading co-workers in creating mainly new products for time series forecasting. I enjoying new challenges and I am capable of leading and inspiring people to work on difficult topics.

\section*{Education}
\begin{tabular}{L!{\VRule}R}
  2016--current&{\bf MSc in Electrical Engineering, Universidade de São Paulo, Brazil.}\\[5pt]
  &{\textbf{Thesis}: Deep Learning: Inspecting Restricted Boltzman Machines} \\
  &{\textbf{Supervisor}: Dr. Prof. Emilio Del Moral Hernandez} \\
  &{\textbf{Concentration}: Computer Science, Scientific Programmind and Data Science.} \\ \\
  
  2016--interrupted&{\bf MSc in Geophysics, Universidade de Campinas, Brazil}.\\[5pt]
  &{\textbf{Thesis}: Global Optimization for AVO inversion in FORTRAN.}\\
  &{\textbf{Supervisors}: Dr. Joerg Schleicher and Dr. Fred Hilterman (UH)}\\
  &{\textbf{Concentration}: Scientific Programming}\\ \\

  2010--2015&{\bf Bachelor of Science in Geophysics, Universidade de São Paulo, Brazil}\\[5pt]
  &{\textbf{Thesis}: Global Optimization for AVO inversion in unconsolidated sediments.} \\
  &{\textbf{Supervisors}: Dr. Liliana Diogo and Dr. Fred Hilterman (UH)}\\
  &{\textbf{Concentration}: Scientific Programming}\\ \\

  2014--2015&{\bf Exchange Program, Exploration Geophysics, University of Houston, US}\\[5pt]
  &{\textbf{Thesis}: Synthetic Seismogram modeling package in MATLAB}\\
  &{\textbf{Supervisor}: Dr. Fred Hilterman}\\
  &{\textbf{Concentration}: Scientific Programming}\\ \\
  
\end{tabular}

\section*{Programming Skills and Projects}
\begin{tabular}{L!{\VRule}R}
  Python Libraries& Tensorflow, Keras, Sklearn, Pandas, Numpy, Multiprocessing, Pyramid, Pytest, Unitest, SQLalchemy \\
  Python Projects& arxivML, ConvNets, pandas (small contributions) \\
  MongoDB& Pymongo, Pymodm, Mongoengine, Mongomock, Mongodb deploy. \\
  Fortran Projects& Development of Genetic Algorithm for Geophysics Applications \\
  ELisp Projects& Maintainer of helm-spotify-plus, python-experiment and emacs.d settings\\
  Misc& SML, Lisp, C++, C, Javascript, MATLAB, LaTeX, Markdown. \\
  Git& Maintainer of internal projects at work.\\
  OS& Unix advanced expertise, OSX and basic Windows.\\
  Amazon AWS& RDS and EC2 deployment (trying to deploy a cluster for mongodb) \\
  Misc Projects& Involved in qutebrowser (keyboard-driven web browser), fish-shell and archlinux forums for some months.
\end{tabular}

\section*{Professional Experience}
\begin{tabular}{L!{\VRule}R}
2016--today&{\bf Analyst in Mathematical Modeling at TEVEC  Metodologias e Sistemas.}\\
&Member of the research team to provide solutions in Machine Learning to attend supply chain problems. The main activity is related to development of new algorithms like clustering, decision tress and mainly neural networks of different architectures to perform time-series predictions. \\[5pt]

2015--2015&{\bf Teacher Assistant at Universidade de São Paulo.}\\
& Supervising students in seismic classes, helping them with theoretical questions and practical exercises in the Seismic Unix package written in C and shell scripts.
\\[5pt]

2013-2013&{\bf Trainee in Geophysics at Institute of Technological Research of São Paulo, IPT}\\
& Performed seismic, bathymetric, sonar data acquisition in Shallow Waters. Such as sonar data processing and interpretation.\\[5pt]

2012-2013&{\bf Teacher Assistant at Universidade de São Paulo}\\
& Supervising students with Physics I, mainly with questions about mechanics.
\end{tabular}

\section*{Scholarships and Awards}
\begin{tabular}{L!{\VRule}R}
  2015& SEG/ExxonMobil Student Education Program - SEP\\
  2015& CNPq - Undergraduate Scholarship\\
  2013--2014& CAPES - Brazilian Mobility Program Scholarship\\
  2011-2012& CNPq - Undergraduate Scholarship\\
  2010-2011& USP Rectory Scholarship for Undergraduate.
\end{tabular}


\bibliographystyle{plain}
\nobibliography{publications.bib}
\section*{Publications and Abstracts}
\begin{itemize}
\item \bibentry{eage2016-avo}
\item \bibentry{eage2016-lockin}
\item \bibentry{sbgf2013-hellbig}
\item \bibentry{agu2012-thermo}
\item \bibentry{agu2012-serena}
\end{itemize}

\section*{Additional Informations}
Things I have been interesting on:

\begin{itemize}
\item Coursera Courses (finished and unfinished ones):
  \begin{itemize}
  \item Amazing course about basic concepts of programming languages focusing in functional programming:
    \textbf{\href{https://www.coursera.org/learn/programming-languages/home/welcome}{Programming Languages, Part A}}
  \item \textbf{\href{https://www.coursera.org/learn/machine-learning/home/welcome}{Machine Learning}}
  \item \textbf{\href{https://www.coursera.org/learn/algorithmic-toolbox/home/welcome}{Algorithmic Toolbox}}
  \item \textbf{\href{https://www.coursera.org/learn/aprender}{Aprendendo a aprender}}
  \end{itemize}
    \item \textbf{\href{https://courses.edx.org/courses/course-v1:CaltechX+CS_1156x+3T2016/info}{Learning from Data}}
    \item \textbf{\href{https://ocw.mit.edu/courses/electrical-engineering-and-computer-science/6-001-structure-and-interpretation-of-computer-programs-spring-2005/}{Structure and Interpretation of computer programs}}
    \item Emacs:
      \begin{itemize}
      \item Talk to me through IRC (channels: #sptk #emacs): bartuka (mynickname)
      \item \textbf{\href{https://www.emacswiki.org/emacs/EmacsWiki}{EmacsWiki}}
      \item \textbf{\href{https://github.com/emacs-tw/awesome-emacs}{Awesome Emacs}}
      \end{itemize}
\end{itemize}

\end{document}
